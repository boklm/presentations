\documentclass{beamer}
\begin{document}
\title{Some good practices for software installation}
\author[Nicolas Vigier]{Nicolas Vigier\\\small{boklm@mars-attacks.org}}
\date{2013-07-10}

\frame{\titlepage}

\frame{\tableofcontents}

\section{Why you should use packages}

\begin{frame}
  \frametitle{Installing software}
  \begin{block}{Simple case: package is available in your distribution}
    \begin{itemize}
      \item apt-get install software
      \item yum install software
      \item urpmi software
    \end{itemize}
  \end{block}

  \begin{block}{A package is not available in your distribution}
    You need to build it yourself
  \end{block}
\end{frame}

\begin{frame}
  \frametitle{Building software}

  \begin{block}{Manual build: ./configure; make; make install}
    \begin{itemize}
      \item You don't need to learn packaging
      \item But creates a lot of problems
    \end{itemize}
  \end{block}

  \begin{block}{Create a package}
    \begin{itemize}
      \item You need to learn packaging
      \item It takes more time the first time
      \item But solve a lot of problems
    \end{itemize}
  \end{block}
\end{frame}

\begin{frame}
  \frametitle{Manual build: dependencies}

  Two possible ways to do it

  \begin{block}{Build on production machines}
    Installing build tools and dependencies on productions machines: not a
    good idea
  \end{block}

  \begin{block}{Use a dedicated build server}
    Need to track the dependencies list to install them on production machines
  \end{block}

  \begin{block}{In both case}
    Need to track build depencies: missing dependency can silently disable
    a feature
  \end{block}
\end{frame}

\begin{frame}
  \frametitle{Manual build: updating and other problems}

  \begin{itemize}
    \item remember build options and build instructions from previous version
    \item remember if you used any patch
    \item files installed by a previous version are not removed on update
    \item uninstall is not easy
  \end{itemize}
\end{frame}

\begin{frame}
  \frametitle{Packages: dependencies management}
  \begin{itemize}
    \item Manual runtime and build dependencies listing
    \item Automatic runtime dependencies detection at the end of package
          build
  \end{itemize}
\end{frame}

\begin{frame}[fragile]
  \frametitle{Packages: updating software}
  \begin{block}{Updating to a new version}
    In many cases, updating a package to a new version is easy
    \begin{enumerate}
      \item Download new version source tarball
      \item Change the version number in the package file
      \item Rebuild the package
    \end{enumerate}
  \end{block}

  \begin{block}{Example with RPM}
\begin{verbatim}
$ cd rpm
$ wget -p SOURCES/ http://something.com/mysoft-1.1.tar.gz
$ sed -i 's/^Version: 1.0/Version: 1.1/' SPECS/mypkg.spec
$ rpmbuild -ba SPECS/mypkg.spec
\end{verbatim}
  \end{block}
\end{frame}

\begin{frame}[fragile]
  \frametitle{Packages: Rebuild with a different option}
  Rebuild a package with different build options is easy
  \begin{enumerate}
    \item Edit the build script to use a different option: the \%build
          option for an rpm .spec file, the debian/build file for a debian
          package
    \item Update the package release number
    \item Rebuild the package
  \end{enumerate}

  \begin{block}{Example with RPM}
\begin{verbatim}
$ cd rpm
$ sed -i 's/--without-option/--with-option/' \
                        SPECS/mypkg.spec
$ rpmbuild -ba SPECS/mypkg.spec
\end{verbatim}
  \end{block}
\end{frame}

\begin{frame}
  \frametitle{Packages: patching software}

  Sometimes patching software is needed

  \begin{itemize}
    \item packages make it easy to include patches
    \item keeping patch separate from original source code
  \end{itemize}
\end{frame}

\begin{frame}
  \frametitle{Packages: installed version}
  Packages make it easy to :
  \begin{itemize}
    \item Check the version of installed package (rpm -q packagename)
    \item Check package providing a file (rpm -qf /path)
    \item Check files provided by a package (rpm -ql packagename)
    \item Check modified files (rpm -q -V packagename)
  \end{itemize}
\end{frame}

\begin{frame}
  \frametitle{Packages: Changelog}
  In case of problem, it's useful to view recent changes on your software
  \begin{itemize}
    \item package changelog
    \item keeping source package in a VCS
  \end{itemize}
\end{frame}

\begin{frame}
  \frametitle{Packages: distribute}

  Create a package repository to distribute packages : a directory
  containing packages and metadata to allow package manager to find and
  download needed packages

  \begin{itemize}
    \item repository shared using http, ftp, rsync, nfs, etc ...
    \item automatically install latest version of packages available
  \end{itemize}
\end{frame}

\begin{frame}
  \frametitle{Packages: easy and automated installation}
  
  \begin{itemize}
    \item Install packages with a simple command
    \item Makes it possible to use configuration management tool
  \end{itemize}
\end{frame}

\section{Why you need a build system}

\begin{frame}
  \frametitle{Packages build system}

  What is a packages build system ?

  \begin{itemize}
    \item simple interface to submit build of a package
    \item manage packages repository
  \end{itemize}
\end{frame}

\begin{frame}
  \frametitle{Build system: easier and faster}

  \begin{itemize}
    \item Submit a build with a single command
    \item Building on fast servers
  \end{itemize}
\end{frame}

\begin{frame}
  \frametitle{Build system: clean environement}

  Each build is done in a new clean environement
  \begin{itemize}
    \item more chances of having a reproductible build
    \item does not depend on custom settings on the build environement
    \item missing build dependencies in source package are detected
  \end{itemize}
\end{frame}

\begin{frame}
  \frametitle{Build system: multiple architectures and distributions}

  A build system can allow you to build packages for :
  \begin{itemize}
    \item different distributions, or releases of a distribution
    \item different architectures (i586, x86\_64, arm, etc ...)
  \end{itemize}
\end{frame}

\begin{frame}
  \frametitle{Build system: less error-prone}

  Moving packages to their repository is a boring task
\end{frame}

\begin{frame}
  \frametitle{Build system: packaging policies}

  Build system can enforce some packaging policies

  \begin{itemize}
    \item rpmlint
    \item lintian
  \end{itemize}
\end{frame}

\begin{frame}
  \frametitle{Build system: permissions}

  Define specific permissions on some repositories or packages
\end{frame}

\begin{frame}
  \frametitle{Build system: VCS integration}
  
  \begin{itemize}
    \item integration with a VCS
    \item traceability of source packages
  \end{itemize}
\end{frame}

\begin{frame}
  \frametitle{How to install a build system ?}

  Some build systems :
  \begin{itemize}
    \item Open Build Service
    \item Mageia Build System
    \item Fedora Build System
    \item Debian Build System
    \item ...
  \end{itemize}
\end{frame}

\section{Why you need a configuration management tool}

\begin{frame}
  \frametitle{Configuration management tool}
  Instead of making changes on the server, describe the change and
  let the configuration management tool do it.

  Some configuration management tools :
  \begin{itemize}
    \item Puppet
    \item CFEngine
    \item Ansible
    \item Salt Stack
  \end{itemize}
\end{frame}

\begin{frame}
  \frametitle{Why use a configuration management tool}

  \begin{itemize}
    \item Manage your infrastructure like a software project, and receive
          contributions
    \item Commit logs : review changes, know what other people are doing
    \item Patch review, patch submissions
    \item Testing environment : easily install a new server for testing
    \item Reusability : create a module to configure your software, and use
          it in different servers or projects
  \end{itemize}
\end{frame}

\begin{frame}
  \begin{center}
    Questions
  \end{center}
\end{frame}

\end{document}
% vim: sw=2
